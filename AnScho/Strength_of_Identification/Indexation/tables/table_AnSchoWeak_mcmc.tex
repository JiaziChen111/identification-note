\documentclass[a4paper,10pt]{article}
\usepackage{amsmath,amsfonts}
\usepackage[left=1cm,right=1cm,top=1cm,bottom=1cm]{geometry}
\usepackage{longtable}
\usepackage{booktabs}
\begin{document}
\centering
\begin{longtable}{ccccccccccccccc}
\toprule
 & RA & PA & GAMQ & TAU & NU & PSIP & PSIY & RHOR & RHOG & RHOZ & SIGR & SIGG & SIGZ & IOTAP \\
\midrule
100 & 0.061 & 0.114 & 0.451 & 0.065 & 47.761 & 0.280 & 0.991 & 7.112 & 3.136 & 20.011 & 38.531 & 4.058 & 4.586 & 2.998 \\
300 & 0.035 & 0.064 & 0.232 & 0.050 & 41.508 & 0.110 & 0.443 & 5.540 & 7.330 & 16.998 & 33.627 & 5.341 & 5.440 & 2.599 \\
900 & 0.020 & 0.035 & 0.148 & 0.029 & 26.825 & 0.063 & 0.341 & 5.555 & 8.928 & 18.567 & 40.908 & 5.454 & 4.947 & 2.533 \\
2700 & 0.018 & 0.030 & 0.122 & 0.033 & 27.352 & 0.027 & 0.182 & 5.442 & 8.287 & 21.504 & 40.337 & 5.572 & 6.773 & 2.613 \\
8100 & 0.018 & 0.032 & 0.129 & 0.033 & 28.766 & 0.023 & 0.154 & 5.070 & 8.175 & 22.077 & 39.142 & 5.596 & 6.651 & 2.458 \\
\bottomrule
\caption{Bayesian Weak Identification An Schorfheide mcmc method}
\label{table:tbl:WeakAnScho_mcmc}
\end{longtable}
\end{document}
