\documentclass[a4paper,10pt]{article}
\usepackage{amsmath,amsfonts}
\usepackage[left=1cm,right=1cm,top=1cm,bottom=1cm]{geometry}
\usepackage{longtable}
\usepackage{booktabs}
\begin{document}
\centering
\begin{longtable}{ccccccccccccccc}
\toprule
 & RA & PA & GAMQ & TAU & NU & PSIP & PSIY & RHOR & RHOG & RHOZ & SIGR & SIGG & SIGZ & IOTAP \\
\midrule
300/100 & 1.738 & 1.691 & 1.541 & 2.294 & 2.607 & 1.177 & 1.341 & 2.337 & 7.011 & 2.548 & 2.618 & 3.949 & 3.558 & 2.601 \\
900/300 & 1.719 & 1.642 & 1.919 & 1.721 & 1.939 & 1.716 & 2.313 & 3.008 & 3.654 & 3.277 & 3.650 & 3.064 & 2.728 & 2.924 \\
2700/900 & 2.664 & 2.580 & 2.466 & 3.425 & 3.059 & 1.307 & 1.602 & 2.939 & 2.785 & 3.475 & 2.958 & 3.065 & 4.107 & 3.095 \\
8100/2700 & 3.051 & 3.140 & 3.177 & 2.992 & 3.155 & 2.509 & 2.531 & 2.795 & 2.960 & 3.080 & 2.911 & 3.013 & 2.946 & 2.822 \\
\bottomrule
\caption{Bayesian Weak Identification An Schorfheide Convergence Ratiosmcmc method}
\label{table:tbl:WeakAnSchoConvergenceRatios_mcmc}
\end{longtable}
\end{document}
