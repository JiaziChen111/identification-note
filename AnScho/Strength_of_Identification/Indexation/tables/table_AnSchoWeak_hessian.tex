\documentclass[a4paper,10pt]{article}
\usepackage{amsmath,amsfonts}
\usepackage[left=1cm,right=1cm,top=1cm,bottom=1cm]{geometry}
\usepackage{longtable}
\usepackage{booktabs}
\begin{document}
\centering
\begin{longtable}{ccccccccccccccc}
\toprule
 & RA & PA & GAMQ & TAU & NU & PSIP & PSIY & RHOR & RHOG & RHOZ & SIGR & SIGG & SIGZ & IOTAP \\
\midrule
100 & 0.074 & 0.043 & 0.508 & 0.085 & 13.694 & 0.570 & 0.309 & 11.928 & 3.445 & 35.078 & 34.940 & 4.233 & 6.501 & 1.117 \\
300 & 0.042 & 0.020 & 0.206 & 0.056 & 11.408 & 0.308 & 0.116 & 8.054 & 9.164 & 23.543 & 29.618 & 5.265 & 7.257 & 0.643 \\
900 & 0.589 & 0.151 & 0.898 & 0.116 & 12.563 & 0.422 & 0.355 & 11.226 & 8.749 & 29.575 & 42.739 & 5.907 & 9.062 & 1.637 \\
2700 & 0.722 & 0.247 & 1.311 & 0.482 & 23.774 & 0.762 & 0.340 & 12.340 & 7.665 & 40.030 & 39.718 & 5.634 & 10.613 & 0.870 \\
8100 & 0.597 & 0.688 & 2.754 & 0.140 & 11.109 & 0.890 & 0.970 & 11.995 & 9.372 & 39.582 & 42.453 & 6.203 & 16.916 & 0.864 \\
\bottomrule
\caption{Bayesian Weak Identification An Schorfheide hessian method}
\label{table:tbl:WeakAnScho_hessian}
\end{longtable}
\end{document}
