\documentclass[a4paper,10pt]{article}
\usepackage{amsmath,amsfonts}
\usepackage[left=1cm,right=1cm,top=1cm,bottom=1cm]{geometry}
\usepackage{longtable}
\usepackage{booktabs}
\begin{document}
\centering
\begin{longtable}{ccccccccccccccc}
\toprule
 & RA & PA & GAMQ & TAU & NU & PSIP & PSIY & RHOR & RHOG & RHOZ & SIGR & SIGG & SIGZ & IOTAP \\
\midrule
300/100 & 2.012 & 1.943 & 1.828 & 2.068 & 2.281 & 1.238 & 1.606 & 2.116 & 8.906 & 2.394 & 2.508 & 4.619 & 3.479 & 2.663 \\
900/300 & 71.936 & 51.269 & 14.988 & 24.005 & 13.497 & 5.182 & 5.927 & 3.548 & 3.362 & 3.829 & 3.414 & 2.967 & 5.882 & 4.442 \\
2700/900 & 1.435 & 1.054 & 1.768 & 1.595 & 2.301 & 4.298 & 3.760 & 4.235 & 3.441 & 6.598 & 3.421 & 3.217 & 2.569 & 2.265 \\
8100/2700 & 3.975 & 7.643 & 11.612 & 27.759 & 6.258 & 8.716 & 3.949 & 2.097 & 2.294 & 2.071 & 2.841 & 3.093 & 5.249 & 5.064 \\
\bottomrule
\caption{Bayesian Weak Identification An Schorfheide Convergence Ratioshessian method}
\label{table:tbl:WeakAnSchoConvergenceRatios_hessian}
\end{longtable}
\end{document}
