\documentclass[a4paper,10pt]{article}
\usepackage{amsmath,amsfonts}
\usepackage[left=1cm,right=1cm,top=1cm,bottom=1cm]{geometry}
\usepackage{longtable}
\usepackage{booktabs}
\begin{document}
\centering
\begin{longtable}{ccccccccccccccc}
\toprule
 & RA & PA & GAMQ & TAU & NU & PSIP & PSIY & RHOR & RHOG & RHOZ & SIGR & SIGG & SIGZ & IOTAP \\
\midrule
300/100 & 1.693 & 1.388 & 1.219 & 1.990 & 2.499 & 1.621 & 1.125 & 2.026 & 7.980 & 2.013 & 2.543 & 3.731 & 3.349 & 1.727 \\
900/300 & 42.405 & 22.690 & 13.054 & 6.180 & 3.304 & 4.103 & 9.188 & 4.182 & 2.864 & 3.769 & 4.329 & 3.366 & 3.746 & 7.640 \\
2700/900 & 3.678 & 4.919 & 4.378 & 12.482 & 5.677 & 5.424 & 2.875 & 3.298 & 2.628 & 4.061 & 2.788 & 2.861 & 3.513 & 1.594 \\
8100/2700 & 2.481 & 8.351 & 6.304 & 0.871 & 1.402 & 3.501 & 8.559 & 2.916 & 3.668 & 2.966 & 3.207 & 3.303 & 4.782 & 2.978 \\
\bottomrule
\caption{Bayesian Weak Identification An Schorfheide Convergence Ratioshessian method}
\label{table:tbl:WeakAnSchoConvergenceRatios_hessian}
\end{longtable}
\end{document}
