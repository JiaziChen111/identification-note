\documentclass[a4paper,10pt]{article}
\usepackage{amsmath,amsfonts}
\usepackage[left=1cm,right=1cm,top=1cm,bottom=1cm]{geometry}
\usepackage{longtable}
\usepackage{booktabs}
\begin{document}
\centering
\begin{longtable}{cccccccccccccc}
\toprule
 & RA & PA & GAMQ & TAU & NU & PSIP & PSIY & RHOR & RHOG & RHOZ & SIGR & SIGG & SIGZ \\
\midrule
300/100 & 1.818 & 1.727 & 1.511 & 1.720 & 2.786 & 1.172 & 0.867 & 2.199 & 8.781 & 2.061 & 2.678 & 3.838 & 2.882 \\
900/300 & 46.337 & 50.758 & 32.420 & 16.347 & 8.046 & 8.402 & 14.011 & 4.162 & 2.779 & 3.655 & 3.824 & 3.052 & 3.508 \\
2700/900 & 0.098 & 0.082 & 0.142 & 0.326 & 0.657 & 0.225 & 0.080 & 0.807 & 2.988 & 3.063 & 1.892 & 3.055 & 2.596 \\
8100/2700 & 138.615 & 128.972 & 58.850 & 23.323 & 12.260 & 57.773 & 231.416 & 15.947 & 2.696 & 6.495 & 5.059 & 4.001 & 5.037 \\
\bottomrule
\caption{Bayesian Weak Identification An Schorfheide Convergence Ratioshessian method}
\label{table:tbl:WeakAnSchoConvergenceRatios_hessian}
\end{longtable}
\end{document}
