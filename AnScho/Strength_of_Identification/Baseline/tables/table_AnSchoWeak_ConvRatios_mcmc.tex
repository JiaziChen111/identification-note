\documentclass[a4paper,10pt]{article}
\usepackage{amsmath,amsfonts}
\usepackage[left=1cm,right=1cm,top=1cm,bottom=1cm]{geometry}
\usepackage{longtable}
\usepackage{booktabs}
\begin{document}
\centering
\begin{longtable}{cccccccccccccc}
\toprule
 & RA & PA & GAMQ & TAU & NU & PSIP & PSIY & RHOR & RHOG & RHOZ & SIGR & SIGG & SIGZ \\
\midrule
300/100 & 1.664 & 1.619 & 1.486 & 2.190 & 2.537 & 0.980 & 1.214 & 2.579 & 7.339 & 1.969 & 2.699 & 3.990 & 2.746 \\
900/300 & 1.806 & 1.747 & 2.034 & 2.045 & 2.269 & 1.262 & 1.900 & 3.015 & 3.285 & 2.954 & 3.566 & 3.145 & 3.280 \\
2700/900 & 2.685 & 2.650 & 2.534 & 3.242 & 3.178 & 0.940 & 1.089 & 3.048 & 2.766 & 3.257 & 3.116 & 2.991 & 3.951 \\
8100/2700 & 2.618 & 2.709 & 2.767 & 3.050 & 3.197 & 1.237 & 1.233 & 2.768 & 2.956 & 3.052 & 2.901 & 3.017 & 3.267 \\
\bottomrule
\caption{Bayesian Weak Identification An Schorfheide Convergence Ratiosmcmc method}
\label{table:tbl:WeakAnSchoConvergenceRatios_mcmc}
\end{longtable}
\end{document}
