\documentclass[a4paper,10pt]{article}
\usepackage{amsmath,amsfonts}
\usepackage[left=1cm,right=1cm,top=1cm,bottom=1cm]{geometry}
\usepackage{longtable}
\usepackage{booktabs}
\begin{document}
\centering
\begin{longtable}{cccccccccccccc}
\toprule
 & RA & PA & GAMQ & TAU & NU & PSIP & PSIY & RHOR & RHOG & RHOZ & SIGR & SIGG & SIGZ \\
\midrule
100 & 0.061 & 0.270 & 0.461 & 0.068 & 38.082 & 0.228 & 2.818 & 5.493 & 3.363 & 29.575 & 40.687 & 3.981 & 11.028 \\
300 & 0.034 & 0.146 & 0.228 & 0.049 & 32.200 & 0.075 & 1.141 & 4.722 & 8.228 & 19.407 & 36.607 & 5.294 & 10.092 \\
900 & 0.020 & 0.085 & 0.155 & 0.034 & 24.358 & 0.031 & 0.722 & 4.746 & 9.009 & 19.107 & 43.514 & 5.551 & 11.034 \\
2700 & 0.018 & 0.075 & 0.131 & 0.036 & 25.799 & 0.010 & 0.262 & 4.821 & 8.305 & 20.746 & 45.198 & 5.535 & 14.533 \\
8100 & 0.016 & 0.068 & 0.121 & 0.037 & 27.490 & 0.004 & 0.108 & 4.449 & 8.183 & 21.107 & 43.708 & 5.567 & 15.825 \\
\bottomrule
\caption{Bayesian Weak Identification An Schorfheide mcmc method}
\label{table:tbl:WeakAnScho_mcmc}
\end{longtable}
\end{document}
