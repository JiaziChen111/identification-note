\documentclass[a4paper,10pt]{article}
\usepackage{amsmath,amsfonts}
\usepackage[left=1cm,right=1cm,top=1cm,bottom=1cm]{geometry}
\usepackage{longtable}
\usepackage{booktabs}
\begin{document}
\centering
\begin{longtable}{cccccccccccccc}
\toprule
 & RA & PA & GAMQ & TAU & NU & PSIP & PSIY & RHOR & RHOG & RHOZ & SIGR & SIGG & SIGZ \\
\midrule
100 & 0.061 & 0.253 & 0.434 & 0.064 & 37.017 & 0.234 & 2.824 & 5.323 & 3.120 & 31.317 & 39.498 & 3.577 & 10.108 \\
300 & 0.036 & 0.160 & 0.250 & 0.050 & 32.281 & 0.081 & 1.309 & 5.108 & 8.286 & 21.856 & 33.807 & 5.079 & 9.340 \\
900 & 0.537 & 2.741 & 1.627 & 0.453 & 186.530 & 0.115 & 1.259 & 8.548 & 7.659 & 35.939 & 42.211 & 5.477 & 10.417 \\
2700 & 0.023 & 0.097 & 0.172 & 0.040 & 29.209 & 0.012 & 0.303 & 4.873 & 8.143 & 21.429 & 43.992 & 5.829 & 16.108 \\
8100 & 1.400 & 4.724 & 2.120 & 1.338 & 238.539 & 0.181 & 1.898 & 4.307 & 9.456 & 26.863 & 49.324 & 6.401 & 21.931 \\
\bottomrule
\caption{Bayesian Weak Identification An Schorfheide hessian method}
\label{table:tbl:WeakAnScho_hessian}
\end{longtable}
\end{document}
