\documentclass[a4paper,10pt]{article}
\usepackage{amsmath,amsfonts}
\usepackage[left=1cm,right=1cm,top=1cm,bottom=1cm]{geometry}
\usepackage{longtable}
\usepackage{booktabs}
\begin{document}
\centering
\begin{longtable}{cccccccccccccccc}
\toprule
 & RA & PA & GAMQ & TAU & NU & PSIP & PSIY & RHOR & RHOG & RHOZ & SIGR & SIGG & SIGZ & RHOZETA & SIGZETA \\
\midrule
300/100 & 1.548 & 1.872 & 1.527 & 1.316 & 1.434 & 1.028 & 1.696 & 3.545 & 5.057 & 3.137 & 2.916 & 8.726 & 2.858 & 0.917 & 2.595 \\
900/300 & 48.525 & 23.078 & 10.612 & 26.082 & 20.119 & 51.954 & 13.376 & 5.179 & 2.441 & 6.008 & 3.867 & 0.808 & 4.287 & 23.943 & 0.896 \\
2700/900 & 6.365 & 12.507 & 10.175 & 8.522 & 3.793 & 1.782 & 1.053 & 1.358 & 8.888 & 1.657 & 3.123 & 16.541 & 3.430 & 4.379 & 27.152 \\
8100/2700 & 1.409 & 2.209 & 2.349 & 6.038 & 3.694 & 2.634 & 7.660 & 6.669 & 2.240 & 6.317 & 3.026 & 1.576 & 3.449 & 2.952 & 8.370 \\
\bottomrule
\caption{Bayesian Weak Identification An Schorfheide Convergence Ratioshessian method}
\label{table:tbl:WeakAnSchoConvergenceRatios_hessian}
\end{longtable}
\end{document}
