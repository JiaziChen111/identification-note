\documentclass[a4paper,10pt]{article}
\usepackage{amsmath,amsfonts}
\usepackage[left=1cm,right=1cm,top=1cm,bottom=1cm]{geometry}
\usepackage{longtable}
\usepackage{booktabs}
\begin{document}
\centering
\begin{longtable}{cccccccccccccccc}
\toprule
 & RA & PA & GAMQ & TAU & NU & PSIP & PSIY & RHOR & RHOG & RHOZ & SIGR & SIGG & SIGZ & RHOZETA & SIGZETA \\
\midrule
100 & 0.065 & 0.252 & 0.456 & 0.079 & 62.196 & 0.199 & 1.713 & 4.129 & 2.481 & 19.910 & 34.959 & 1.200 & 10.716 & 0.290 & 0.246 \\
300 & 0.034 & 0.157 & 0.232 & 0.035 & 29.722 & 0.068 & 0.969 & 4.879 & 4.182 & 20.818 & 33.982 & 3.489 & 10.210 & 0.089 & 0.213 \\
900 & 0.546 & 1.211 & 0.822 & 0.302 & 199.325 & 1.181 & 4.319 & 8.423 & 3.403 & 41.694 & 43.800 & 0.940 & 14.590 & 0.707 & 0.064 \\
2700 & 1.158 & 5.049 & 2.787 & 0.859 & 251.987 & 0.702 & 1.516 & 3.813 & 10.081 & 23.025 & 45.594 & 5.183 & 16.679 & 1.032 & 0.576 \\
8100 & 0.544 & 3.717 & 2.183 & 1.729 & 310.263 & 0.616 & 3.872 & 8.476 & 7.528 & 48.487 & 45.992 & 2.723 & 19.175 & 1.015 & 1.607 \\
\bottomrule
\caption{Bayesian Weak Identification An Schorfheide hessian method}
\label{table:tbl:WeakAnScho_hessian}
\end{longtable}
\end{document}
