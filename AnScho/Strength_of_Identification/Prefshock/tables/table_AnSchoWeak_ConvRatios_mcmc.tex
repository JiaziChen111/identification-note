\documentclass[a4paper,10pt]{article}
\usepackage{amsmath,amsfonts}
\usepackage[left=1cm,right=1cm,top=1cm,bottom=1cm]{geometry}
\usepackage{longtable}
\usepackage{booktabs}
\begin{document}
\centering
\begin{longtable}{cccccccccccccccc}
\toprule
 & RA & PA & GAMQ & TAU & NU & PSIP & PSIY & RHOR & RHOG & RHOZ & SIGR & SIGG & SIGZ & RHOZETA & SIGZETA \\
\midrule
300/100 & 1.605 & 1.759 & 1.504 & 1.374 & 1.568 & 1.048 & 1.581 & 3.289 & 5.427 & 2.911 & 3.266 & 5.257 & 2.720 & 0.934 & 2.129 \\
900/300 & 1.951 & 1.752 & 1.883 & 2.866 & 2.617 & 1.083 & 1.728 & 3.197 & 3.252 & 3.094 & 3.418 & 1.762 & 3.331 & 1.092 & 1.176 \\
2700/900 & 3.073 & 2.805 & 3.029 & 2.696 & 2.455 & 1.088 & 1.325 & 2.595 & 4.810 & 2.729 & 3.096 & 2.044 & 3.042 & 1.223 & 1.764 \\
8100/2700 & 2.289 & 2.444 & 2.307 & 3.209 & 3.534 & 0.924 & 0.925 & 3.382 & 3.357 & 3.509 & 3.069 & 3.251 & 3.933 & 1.233 & 1.616 \\
\bottomrule
\caption{Bayesian Weak Identification An Schorfheide Convergence Ratiosmcmc method}
\label{table:tbl:WeakAnSchoConvergenceRatios_mcmc}
\end{longtable}
\end{document}
