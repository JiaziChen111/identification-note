\documentclass[a4paper,10pt]{article}
\usepackage{amsmath,amsfonts}
\usepackage[left=1cm,right=1cm,top=1cm,bottom=1cm]{geometry}
\usepackage{longtable}
\usepackage{booktabs}
\begin{document}
\centering
\begin{longtable}{cccccccccccccc}
\toprule
 & RA & PA & GAMQ & TAU & NU & PSIP & PSIY & RHOR & RHOG & RHOZ & SIGR & SIGG & SIGZ \\
\midrule
300/100 & 1.686 & 1.257 & 1.284 & 2.163 & 2.489 & 1.243 & 1.042 & 2.116 & 8.753 & 1.568 & 2.493 & 2.589 & 1.166 \\
900/300 & 1.735 & 2.191 & 1.922 & 1.237 & 1.607 & 3.519 & 5.187 & 3.378 & 3.176 & 3.096 & 3.604 & 3.202 & 2.603 \\
2700/900 & 2.749 & 2.802 & 2.501 & 2.697 & 2.911 & 2.660 & 2.552 & 2.925 & 2.492 & 3.119 & 3.047 & 3.001 & 2.996 \\
8100/2700 & 2.985 & 2.832 & 3.077 & 2.994 & 3.342 & 3.365 & 3.713 & 2.974 & 2.908 & 2.789 & 2.837 & 2.927 & 2.959 \\
\bottomrule
\caption{Bayesian Weak Identification An Schorfheide Convergence Ratioshessian method}
\label{table:tbl:WeakAnSchoConvergenceRatios_hessian}
\end{longtable}
\end{document}
