\documentclass[a4paper,10pt]{article}
\usepackage{amsmath,amsfonts}
\usepackage[left=1cm,right=1cm,top=1cm,bottom=1cm]{geometry}
\usepackage{longtable}
\usepackage{booktabs}
\begin{document}
\centering
\begin{longtable}{cccccccccccccc}
\toprule
 & RA & PA & GAMQ & TAU & NU & PSIP & PSIY & RHOR & RHOG & RHOZ & SIGR & SIGG & SIGZ \\
\midrule
300/100 & 1.829 & 1.285 & 1.559 & 2.906 & 2.302 & 1.033 & 0.658 & 2.336 & 15.792 & 1.244 & 2.406 & 2.979 & 0.976 \\
900/300 & 1.678 & 2.299 & 1.873 & 1.410 & 1.691 & 2.227 & 4.699 & 3.301 & 3.175 & 3.162 & 3.555 & 3.434 & 2.805 \\
2700/900 & 2.676 & 2.827 & 2.464 & 2.561 & 2.610 & 2.106 & 2.345 & 3.010 & 2.482 & 3.074 & 3.090 & 2.986 & 3.099 \\
8100/2700 & 2.931 & 2.845 & 3.025 & 2.870 & 2.977 & 3.158 & 3.509 & 2.822 & 2.697 & 2.730 & 2.869 & 3.074 & 2.956 \\
\bottomrule
\caption{Bayesian Weak Identification An Schorfheide Convergence Ratioshessian method}
\label{table:tbl:WeakAnSchoConvergenceRatios_hessian}
\end{longtable}
\end{document}
