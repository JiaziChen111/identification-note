\documentclass[a4paper,10pt]{article}
\usepackage{amsmath,amsfonts}
\usepackage[left=1cm,right=1cm,top=1cm,bottom=1cm]{geometry}
\usepackage{longtable}
\usepackage{booktabs}
\begin{document}
\centering
\begin{longtable}{cccccccccccccc}
\toprule
 & RA & PA & GAMQ & TAU & NU & PSIP & PSIY & RHOR & RHOG & RHOZ & SIGR & SIGG & SIGZ \\
\midrule
100 & 0.055 & 0.145 & 0.444 & 0.055 & 40.977 & 0.279 & 2.779 & 4.398 & 3.865 & 34.385 & 40.505 & 3.206 & 9.897 \\
300 & 0.030 & 0.049 & 0.226 & 0.050 & 31.939 & 0.101 & 0.853 & 3.678 & 17.043 & 14.894 & 35.231 & 3.558 & 5.383 \\
900 & 0.017 & 0.046 & 0.145 & 0.024 & 18.887 & 0.075 & 1.748 & 4.164 & 17.572 & 17.036 & 43.902 & 4.465 & 5.656 \\
2700 & 0.016 & 0.045 & 0.122 & 0.022 & 16.903 & 0.054 & 1.471 & 4.197 & 14.471 & 17.612 & 45.664 & 4.491 & 5.992 \\
8100 & 0.015 & 0.043 & 0.125 & 0.021 & 17.104 & 0.058 & 1.778 & 3.940 & 13.030 & 16.057 & 43.692 & 4.628 & 6.020 \\
\bottomrule
\caption{Bayesian Weak Identification An Schorfheide mcmc method}
\label{table:tbl:WeakAnScho_mcmc}
\end{longtable}
\end{document}
