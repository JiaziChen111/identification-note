\documentclass[a4paper,10pt]{article}
\usepackage{amsmath,amsfonts}
\usepackage[left=1cm,right=1cm,top=1cm,bottom=1cm]{geometry}
\usepackage{longtable}
\usepackage{booktabs}
\begin{document}
\centering
\begin{longtable}{cccccccccc}
\toprule
 & ALPHA & RA & DELTA & RHOA & SIGA & THETA & KAPPA & RHOUPSILON & SIGUPSILON \\
\midrule
300/100 & 2.274 & 0.967 & 1.094 & 2.021 & 2.130 & 0.959 & 1.785 & 2.239 & 0.793 \\
900/300 & 1.703 & 1.015 & 2.093 & 2.598 & 2.786 & 1.699 & 2.048 & 2.781 & 1.943 \\
2700/900 & 1.425 & 1.027 & 1.388 & 2.937 & 1.907 & 2.995 & 2.831 & 2.893 & 2.355 \\
8100/2700 & 1.291 & 1.182 & 2.113 & 2.995 & 3.214 & 2.294 & 2.488 & 2.974 & 2.566 \\
\bottomrule
\caption{Bayesian Weak Identification An Schorfheide Convergence Ratioshessian method}
\label{table:tbl:WeakAnSchoConvergenceRatios_hessian}
\end{longtable}
\end{document}
