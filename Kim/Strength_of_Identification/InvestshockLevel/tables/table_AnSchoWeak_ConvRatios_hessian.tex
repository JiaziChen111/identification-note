\documentclass[a4paper,10pt]{article}
\usepackage{amsmath,amsfonts}
\usepackage[left=1cm,right=1cm,top=1cm,bottom=1cm]{geometry}
\usepackage{longtable}
\usepackage{booktabs}
\begin{document}
\centering
\begin{longtable}{cccccccccc}
\toprule
 & ALPHA & RA & DELTA & RHOA & SIGA & THETA & KAPPA & RHOUPSILON & SIGUPSILON \\
\midrule
300/100 & 1.911 & 0.988 & 1.191 & 2.097 & 2.186 & 0.947 & 1.715 & 2.324 & 0.692 \\
900/300 & 1.735 & 1.004 & 2.179 & 2.657 & 2.743 & 1.011 & 1.077 & 2.632 & 1.366 \\
2700/900 & 1.477 & 1.004 & 1.363 & 2.951 & 1.936 & 1.001 & 0.849 & 2.830 & 1.275 \\
8100/2700 & 1.209 & 1.019 & 1.781 & 2.997 & 3.024 & 1.055 & 1.040 & 2.980 & 1.238 \\
\bottomrule
\caption{Bayesian Weak Identification An Schorfheide Convergence Ratioshessian method}
\label{table:tbl:WeakAnSchoConvergenceRatios_hessian}
\end{longtable}
\end{document}
